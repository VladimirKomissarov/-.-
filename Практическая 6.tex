\documentclass[a4paper,12pt]{scrartcl}
\usepackage{mathtext}                    % Русский язык в формулах
\usepackage{cmap}                        % Поддержка поиска русских слов в PDF (pdflatex)
\usepackage[cp1251]{inputenc}            % Выбор языка и кодировки
\usepackage[english, russian]{babel}     % Настройки для русского языка
\usepackage{tikz}
\usepackage[european,cuteinductors,smartlabels]{circuitikz}



\begin{document}
\begin{center}
\title{Практическая работа №6. Электрические схемы}
\title{Студента группы 8871 Комиссарова Владимира}
\end{center}
\begin{center}
\title{Задание 1}
\end{center}
\begin{circuitikz}

\draw (0,-5) to [C, l_={$C_1$},*-*] (5,-5);
\draw (2.5,0) to [R, l={$R_1$},*-*] (5,-5);
\draw (0,-5) to [L, l={$L_1$},*-*] (2.5,0);

\draw (2.5,0) -- (15,0);
\draw (5,-5) -- (13,-5);
\draw (0,-5) -- (0,-7);
\draw (0,-7) -- (17,-7);
\draw (17,-7) -- (17,-5);

\draw (13,-5) to [C, l_={$C_2$},*-*] (15,-2.5);
\draw (15,-2.5) to [R, l={$R_2$},*-*] (15,0);
\draw (15,-2.5)  to [L, l={$L_2$},*-*] (17,-5);

\end{circuitikz}
\begin{center}


\end{document}

